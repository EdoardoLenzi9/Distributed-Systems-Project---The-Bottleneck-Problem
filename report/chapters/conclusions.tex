\chapter{Conclusions}\label{ch:conclusions}

The goal of our work was to develop a system that models the behaviour of a bridge. We managed to ensure
most of the requisites.\\ In particular our model avoids any accident between two or more cars: this is 
really important because in a real world the first thing to ensures is that nobody gets hurt or damaged. 
For the same reason we made the cars able to detect any broken car. The case of the broken engine is the simpliest
to handle because the broken car can still communicate with the others and so notify them she is not moving.
Moreover the tow truck can remove the car so that the rear ones can eventually move again.\\ Similar to that
was ensuring that nobody will wait forever to cross the bridge. In a realistic situation it is not contemplated
that one or more cars could never cross the bridge.\\
The chosen strategy is FIFO order. This avoids starvation but isn't the best choice for performance: consider
the case where one car arrives on the left side while on the right there are, for example, 20 cars, and that the bridge
has capacity equal to 1 and lenght 30. The car on the left has to wait for all the other 20 on the right side: this 
is a very long time due to the fact that only one car per time can cross the bridge and to the lenght of the bridge.
If that car could cross without waiting for all the others its waiting time would decrease significally while
the waiting time for the others will increase only as if there were one car more before them. So one thing 
that can improve the system could be order the cars by minimizing the average waiting time.\\
Finally the system satisfy all the requisites of the bridge (capacity and one way cross).
