\chapter{Conclusions}\label{ch:conclusions}

We have encountered numerous impediments during the development of the project that 
have led to a delay in the delivery. 

In first place we start the development on a single machine but when we try 
to work on different machines we realize that there were some communication problems
that forced us to update the communication architecture.

In second place we find that even in a local network there can be huge delays in message exchange,
so any solution based on timing, instead of event causality, will fail sooner or later; 
this forces us to change the algorithm for the car movements in queue. 

At the same time we realize that both to obtain a more realistic result and, above all, a more consistent result,
many new variables had to be integrated (such as speed and car dimensions); this has impacted a lot
on the movements logic and therefore on tests.

Another major impediment was language, as Erlang is poorly documented outside the official documentation 
and also some functionality of the new $gen\_statem$ behaviour don't work as expected 
(i.e. event \textit{postpone} freezes the FSM and event \textit{timer} is blocking). 

Unfortunately we discover too late the existence of the (relatively) new language Elixir~\cite{23} 
which seems to be more documented and supported by the community 
(if we had known of its existence at the beginning of the project surely we would 
have taken it into consideration as an alternative language). 

Also the implementation of the web service was hard at the beginning cause at first we tried to use Yaws~\cite{24} 
but it didn't satisfy our needs, only after some time we found the Cowboy framework.\\

\noindent
Anyway the goal of our work was to develop a system that models the behaviour 
of a bridge and, finally, we reach it. 

Our attention focuses in first place on avoiding accidents between two or more cars: 
this is really important because in a real world application the first thing to ensures 
is that nobody gets hurt or damaged. 

For the same reason, in the second place, we focus our attention on 
fail detection; this has paved the way for a new idea for an optimization, 
in fact FIFO strategy avoids starvation but isn't the best choice for performance.

Consider the case where one car arrives on the left side while on the right there are, 
for example, 20 cars, and that the bridge has capacity equal to 1 and length 30. 

The car on the left has to wait until the other 20 cars have crossed the bridge; 
this is a very long time due to the fact that only one car per time can cross the bridge 
and to the length of the bridge.

If the lonenly car could cross without waiting for all the others its waiting time would 
decrease significally while the waiting time for the others will increase only 
as if there were one car more before them. 
So one thing that can improve the system could be order the cars by minimizing 
the average waiting time.\\

Another important matter we have to discuss the main bottlenecks
for the system performace. 

Everytime a car has to move it has to send many message to its adjacent cars: 
for this reason the cars cannot move in a fluent way but have to stop sometimes. 

Moreover we cannot visualize a new car if there is already another one on the 
street that has $crash\_type$ 2, in fact, if it is broken it isn't able to tell 
the new car its position and so the new one has to wait until it is removed 
(notice that the tow truck will be called by the new car before
going on the street).

Pay also attention to the $max\_RTT$ that will be set because a short one could led 
to assume that some cars are dead even if they are not 
(only because their response to the check would arrive too late), 
but on the other hand a long one could be unnecessary 
(and so the cars would wait longer before making a decision).\\

The last aspect we would talk about is how the $crash\_type$ effects the performance of 
the system. 

As we could imagine a car with a broken engine would be easier 
to handle rather than a system failure;
this because it can call by itself the tow truck and can also keep sending messages 
to the other and so they can move and reach the last ammisible position without 
crashing with it. In other words the whole system works faster.

In conclusion we couldn't provide a better solution for the above bottlenecks 
cause they depend on the problem nature and must be payed regardless 
of the algorithms and their implementation.