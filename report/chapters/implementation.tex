\chapter{Implementation}\label{ch:implementation}
In this chapter we will explain how the system was implemented, so we will analize the languages, the data structures that were used
and something more, inherent to that.

\section{Language}
To implement the project Erlang language was used. The decision was made due to the nature of the language itself.\\
In fact, Erlang allows to handle concurrency and distributed programming: programs that can handle several threads of execution at the same time. 
Actually each processo only swaps between jobs so fast, so it only gives the illusion of running them all at the same time. 
What is great about Erlang is that it make it easy to create parallel threads of execution and it also permits these
threads (called process because they do not share data) to communicate with each other.\\
Furthermore Erlang can models finite state machine thanks to the $gen_statem$ behaviour and that fitted perfectly
with the nature of the problem.\\ \\
The standard data interchange format used is JSON. The main reason of this decision is that it is used 
for transmitting data between a web application and a server. 
Moreover JSON files are lightweight, text-based and human-readable, so we tought that would be good for our goal.

\section{TODO-- ?docker?}
Details about the implementation: every choice about platforms, languages, software/hardware, middlewares, which has not been decided in the requirements.

\section{Data structures}
As described in chaper 1, cars and bridge must be aware about some of their characteristics.
In order to do that we decided to implement the car with a new data structure as follows:

\subsection{Environment}

To model the environment the following attributes, which are set in environment.js for once, will be used:
\begin{itemize}
    \item $host$:\\ Web service IP:PORT
    \item $max\_speed$:\\ the maximun speed allowed
    \item $bridge\_capacity$: number\\ it defines the maximun number of cars that can cross the bridge
    at the same time
    \item $bridge\_length$: number\\ it defines the lenght of the bridge
    \item $tow\_truck\_time$: number\\ time necessary for the tow truck to arrive and remove the dead car
    \item $max\_RTT$:number\\ it is the maximun RTT that could happen
\end{itemize}

These attributes will never be modified during the simulation.

\subsection{Car}

Each car A is modelled exploiting the following attributes:
\begin{itemize}
    \item $name$: number\\ it identifies univocally the car
    \item $side$: number\\ if equal to 1 means that the car is on the right side, 
    if equal to -1 it means that it is on left side
    \item $power$: number\\ it defines the number of rear and front cars that the car can reach
    \item $size$: number \\ it defines the lenght of the car
    \item $speed$: number\\ it defines the distance that the car goes through in one turn\\
    it has a maximun value specified in \textbf{environment.js}\\
    it is initialized with 0 to prevent crashing 
    \item $position$: number \\it indicates the distance between the car and the bridge or the car and the end of the
    bridge\\it is initialized as $pos(B)+1*side(B)$, where B is the first front car, if B is on the same side of A, or
    as $-pos(B)$ if B is on the other side\\ when negative it means that the car is on the left side, otherwise on the right side
    \item $crossing$: boolean\\ if true it means that the car is on the bridge
    \item $syncrhonized$: boolean\\ if true it means that the car is syncrhonized    
    \item $delta$: number\\ it is the difference between the local time and the global time
    \item $arrival\_time$: number\\ it defines the time when the car has arrived in the queue
    \item $current\_time$: number\\ it indicates the local time at each turn
    \item $adj$: another data structure\\ it contains the lists of front and rear cars
    \item $state$: string\\ it defines the current state of the car\\ it could be \textbf{init}, \textbf{sync}, \textbf{normal} or \textbf{dead}
\end{itemize}
These above are the car metadata, while the following are settings and bridge metadata 
(these are taken from the environment definition but must be part of the car specifics too):
\begin{itemize}
    \item $host$:\\ Web service IP:PORT
    \item $bridge\_capacity$: number\\ it defines the maximun number of cars that can cross the bridge
    at the same time
    \item $bridge\_length$: number\\ it defines the lenght of the bridge
    \item $max\_speed$:\\ the maximun speed allowed
    \item $tow\_truck\_time$: number\\ time necessary for the tow truck to arrive and remove the dead car
    \item $max\_RTT$:number\\ it is the maximun RTT that could happen
\end{itemize}

Note that $bridge\_capacity$ and $bridge\_length$ are imposed by the environment and cannot change during the
simulation. Furthermore $name$ and $power$ are defined independently, while $delta$ and $adj$ 
depend on the other cars.\\

The problem allows to describe the car as a finite state machine. We choose this modality because
the car can switch between four main different states. The first one is a so called \textbf{init}: the 
undefined attributes are computed here. It is assumed that in this state the car cannot receive
any sort of event and that cannot crash. \\
When the car is syncrhonized it changes its state in \textbf{sync}: the car call the web service to receive the list of adiacents cars. 
When it is received the car knows how to communicate with the current front car and so can define its $speed$ and $position$ according to
the state of the front car. Finally thee car can change its state to \textbf{normal}.\\
In \textbf{normal} the car still sends repeatedly a check event to the front car in order to regolate hers speed
(and consequently the new position).\\
In particular, at every iteration, the car computes the following function to determinate the 
new $speed$: \[\frac{pos(A)-pos(B)}{RTT + turn}\] 
and the new $position$: \[pos(A) + \frac{|pos(A)-pos(B)|}{RTT + turn}\]
where $pos(x)$ = $position$ of x. \\
Obviously this means that if the distance between A and B is zero, the car will change its speed to 
zero: this avoids crashing during the next turn even if the front car says that its $speed$ is greater
then zero. In fact, if B crashes immediately after sending its state, the other car assumes that B
is moving and so she decides to move too ending up colliding with B.\\ When the car finally reach the position
zero she become a leader and so its state would be the \textbf{leader} state. If there is a car B on the opposite side 
(that can be in the leader state too), they have to decide who will start crossing:
\begin{itemize}
    \item if A's $crossing$ attribute is true and her arrival time is less than B's,
     she start crossing the bridge
    \item if A's $crossing$ attribute is false and her arrival time is less than B's,
     she change it to true
    \item if A's arrival time is greater than B's, she stops and waits her turn
\end{itemize}

When it's its turn, the car change its $position$, that will be equal to the bridge lenght, and its $speed$,
that will be the maximun speed allowed. Before changing state to normal the car has also to propagate to the first n cars behind her (n is equal to the bridge capacity) 
to change $crossing$ value to true. After that she moves and returns to the \textbf{normal} state.\\
Obviously there is another state called $\textbf{dead}$ that could be reached from every state, except for \textbf{init}, 
when the car's engine or the car's engine and system crash or when the car completes the crossing. 
In the first case the car cannot move anymore and so somebody (the car itself or the first behind her) must call the tow truck to be removed. 
\\Notice that if there is a car whose $speed$ = 0 it may mean that the car is stationary, waiting the front
 car to move, or in \textbf{dead} status.\\

\subsubsection{Communication}

It is important to underline how two cars can send message to each other. \\
First of all, each car has a supervisor and, when a car A want to send something to another car B,
this happens:
\begin{itemize}
   \item[1.] A sends a message to hers supervisor S$_{A}$
   \item [2.] S$_{A}$ calls a process, we can refer at it as $Timer$, which is used to send a timeout
   if the supervisor didn't receive a response from the other within a certain time. In that case S$_{A}$ assumes 
   that B is dead and so A will call a tow truck to remove it.
   \item[3.] S$_{A}$ sends it to the supervisor of B S$_{B}$
   \item [4.] S$_{B}$ sends an event to B which contains the message of A
   \item [5.] B send a message to S$_{B}$ whit the response
   \item [6.] S$_{B}$ sends it to S$_{A}$, going through the $Timer$ as S$_{A}$ has done at point 2
   \item [7.] finally S$_{A}$ sends another event to A with the response of B, to which A reacts appropiately
\end{itemize}

\subsection{Adj}
We used another data structure for the adiacents cars, called $adj$:
\begin{itemize}
    \item $front\_cars$: list of cars\\ it contains all the car ahead that are rechable\\ the first element
     is the next car
    \item $rear\_cars$: list of cars\\ it contains all the car behind that are rechable\\ the last element
     is the previous car
\end{itemize} 