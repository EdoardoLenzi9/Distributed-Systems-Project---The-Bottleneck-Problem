\chapter{Analysis}\label{ch:analysis}

In this section we will discuss the requirements which our system must satisfy 
discriminating them between functional and non functional.

\section{Functional requirements}

The system will be able to:
\begin{itemize}
    \item \textbf{Coordinate the traffic} among the bridge following the rule 
        that it can be crossed only in one way per turn. 
        So it never happens that two cars crossing the bridge at the same time, 
        in opposite ways.
    \item \textbf{Avoid} any kind of \textbf{accident} (also between enqueued cars).
    \item The maximum amount of cars that can cross the bridge at the same time, 
        in the same direction (a \textbf{block}), is equivalent to its capacity $c$.
    \item The decision of who can cross the bridge will be taken by the cars 
        themselves through a communication (\textbf{agreement}) that they 
        instaurate without any human help.
    \item If a car is broken (it can't communicate and/or move any more) 
        it will be removed by a \textbf{tow truck}.
    \item provide a \textbf{UI} in order to monitoring the \textbf{simulation} 
        and manage \textbf{settings}.
    \item generate new cars.
\end{itemize}


\section{Non functional requirements}

We can distinguish the non functional requirements in the following macro areas:
\begin{itemize}
    \item \textbf{Safety}: the system guarantees a correct use of the bridge, avoiding 
        any sort of accident between two or more cars. 
        In fact it ensures that the bridge can be crossed only by cars that are moving 
        in the same direction. 
        Notice that we are not excluding the possibility of a car to break down 
        (the cause of crashing is something that doesn't belong to the task of the system).
    \item \textbf{Security}: (bonus requirement, not necessary) 
        the system avoids man in the middle attacks using HTTPS.
    \item \textbf{Strength}: the system can work also in particular cases such as 
        when a car breaks down and needs to be removed. 
        This situation does not lead to an accident because the other cars can find out 
        if a car is broken.
    \item \textbf{Scalability}: cars, clients and web services. The project requires that
        cars can scale arbitrarily and there is a single web service; 
        anyway our architecture can be easily modified in order to have an arbitrary number 
        of web services (cause we use Mnesia~\cite{1}, a distributed database and cars are 
        bounded to a particular web service dynamically).
    \item \textbf{Consistency}: due to the relative point of view of each process this 
    requirement is not guarantee at all.
    \item \textbf{Starvation}: if a car requires to cross the bridge, 
        its request will be satisfied eventually; 
    it will never happen that a car waits for its turn forever.
    \item \textbf{Fairness}: if a car $A$ has arrived before car $B$, 
        then car $A$ will be the first one who cross the bridge (\textbf{FIFO ordering}).
        Such as consistency there are scenerys in which, due to concurrency, the simulation 
        generates two or more cars in the same instant and the requests arrive 
        at the web service in random order (this isn't a huge issue cause in the reality 
        two cars cannot be spawned in the scene in the same instant).
\end{itemize}